\documentclass[floatfix,ApJL,twocolumn]{aastex631}

\usepackage{amssymb}
\usepackage{amsmath}
\usepackage{microtype}
\usepackage{url}
\usepackage{xspace}
\usepackage{xcolor}
\usepackage{ifxetex}
\ifxetex
\usepackage{fontspec}
\defaultfontfeatures{Extension = .otf}
\fi
\usepackage{fontawesome}



\setlength{\parindent}{3.0ex}


% Projects:
\newcommand{\project}[1]{\textsf{#1}}

\newcommand{\python}{\project{Python}}
\newcommand{\jupyter}{\project{Jupyter}}
\newcommand{\exoplanet}{\project{exoplanet}}
\newcommand{\lightkurve}{\project{lightkurve}}
\newcommand{\starry}{\project{starry}}
\newcommand{\pymc}{\project{PyMC3}}
\newcommand{\pymcextra}{\project{pymc3-ext}}
\newcommand{\celerite}{\project{celerite}}
\newcommand{\astropy}{\project{astropy}}
\newcommand{\scipy}{\project{scipy}}
\newcommand{\jupytext}{\project{jupytext}}
\newcommand{\sphinx}{\project{sphinx}}
\newcommand{\jupyterbook}{\project{Jupyter-book}}
\newcommand{\arviz}{\project{ArviZ}}
\newcommand{\nbconvert}{\project{nbconvert}}
\newcommand{\numpy}{\project{numpy}}
\newcommand{\pandas}{\project{pandas}}
\newcommand{\matplotlib}{\project{matplotlib}}
\newcommand{\corner}{\project{corner}}

\newcommand{\lvk}{\project{LVK}}
\newcommand{\tess}{\project{TESS}}
\newcommand{\mast}{\project{MAST}}
\newcommand{\exofop}{\project{ExoFOP}}


% math
\newcommand{\T}{\ensuremath{\mathrm{T}}}
\newcommand{\dd}{\ensuremath{ \mathrm{d}}}
\newcommand{\unit}[1]{{\ensuremath{ \mathrm{#1}}}}
\newcommand{\bvec}[1]{{\ensuremath{\boldsymbol{#1}}}}


\DeclareMathOperator{\invG}{Inv-\mathnormal{\Gamma}}
\DeclareMathOperator{\N}{\mathcal{N}}
\DeclareMathOperator{\U}{\mathcal{U}}
\DeclareMathOperator{\Un}{\mathcal{U}}
\DeclareMathOperator{\Par}{\mathcal{P}ar}
\DeclareMathOperator{\tmin}{\mathnormal{t_{\rm min}}}
\DeclareMathOperator{\tmax}{\mathnormal{t_{\rm max}}}





%% affiliation shortcuts
\newcommand{\SPA}{School of Physics and Astronomy, Monash University, Clayton VIC 3800, Australia}
\newcommand{\OzGravMonash}{OzGrav: The ARC Centre of Excellence for Gravitational Wave Discovery, Clayton VIC 3800, Australia}
\newcommand{\AMNH}{Department of Astrophysics, American Museum of Natural History, New York, NY 10024, USA}
\newcommand{\CCA}{Center for Computational Astrophysics, Flatiron Institute, New York, NY 10010, USA}
\newcommand{\CUNY}{Graduate Center, City University of New York, 365 5th Avenue, New York, NY 10016, USA}
\newcommand{\BMCC}{Department of Science, BMCC, City University of New York, New York, NY 10007, USA}

\newcommand{\projectUrl}{\href{http://google.com/}{some-link}}
\newcommand{\projectGit}{\href{https://github.com/avivajpeyi/compas_ml_surrogate}{github.com/avivajpeyi/compas_ml_surrogate}}

\newif\ifdraft
\drafttrue % switch to false in non-draft version (thereby hiding the todos)
\draftfalse
\newcommand{\inDraftVersion}[1]{\ifdraft #1\fi}


% TODOs
\newcommand{\todo}[3]{\inDraftVersion{{\color{#2}\emph{#1}: #3}}}
\newcommand{\avi}[1]{\todo{Avi}{red}{#1}}
\newcommand{\alltodo}[1]{\todo{TODO}{red}{#1}}
\newcommand{\citeme}{{\color{red}(citation needed)}}


\newcommand{\red}{\inDraftVersion{\textcolor{red}}}
\newcommand{\textuit}[1]{\textit{\underline{#1}}}


\newcommand{\github}[1]{\href{#1}{\textcolor{gray}{\faGithubSquare}}}




\shorttitle{SFR from CBC Mergers}


\begin{document}

\title{Using COMPAS to learn about the star formation rate from LIGO CBC mergers}

\shortauthors{Vajpeyi et al.}
\author[0000-0002-4146-1132]{Avi Vajpeyi}%
\affiliation{\SPA}
\affiliation{\OzGravMonash}






\begin{abstract}
We present BLAH
\end{abstract}


\keywords{%
  methods: data analysis ---
  methods: statistical ---
  miscellaneous
}


\section{Introduction} \label{sec:intro}

Astrophysical processes that govern stellar, binary system evolution, and the production of binary black holes (BBH) that produce gravitational waves (GW) detectable by current instrumentation (e.g. aLIGO, VIRGO, KAGRA, the \lvk) are very uncertain. 
Over the next few years, we will accumulate a population of several hundred BBH mergers. 
We can use the population of mergers to constrain some parameters describing astrophysical processes, such as the star formation rate in the early universe. 
In this work, we use \compas, a rapid stellar population synthesis tool, to model the universe with different star formation rates (SFR) and try to place constraints on the SFR that can reproduce the \lvk\ detections. 


The remainder of the paper is organised as follows.
Section~\ref{sec:method} outlines...
Results are summarised in Section~\ref{sec:results}.
The data products, and software to reproduce the results available online as supplementary materials (\projectUrl).
Finally, we discuss caveats and provide concluding remarks in Section~\ref{sec:conclusion}.

\section{Method} \label{sec:method}


The \lvk\ has detected $\sim90$ GW signals from a population of merging BBHs. 
Each BBH has some parameters $\vec{\theta}$ that describe the system, such as the BBH's chirp-mass $\mathcal{M}$, spin magnitude $\chi$, spin angle $\theta$, and redshift $z$.
On the other hand, \compas\ can generate populations of merging BBHs using different initial star formation parameters $\vec{\lambda}$. 

Therefore, we can use \compas\ to determine the posterior on $\vec{\lambda}$ given \lvk's $N_{\rm obs}$ observed BBHs and their parameters $\vec{D} = \{ p_i(\vec{\theta}|h), ...\}$ (where $h$ is the strain data):
\begin{equation}
    p(\vec{\lambda}| \vec{D}) \sim \pi(\vec{\lambda})\ \mathcal{L}(\vec{D}| \vec{\lambda})\ .
\end{equation}


Here, $\mathcal{L}(\vec{D}| \vec{\lambda})$ can be written as:
\begin{equation}
    \ln \mathcal{L}(\vec{D}| \vec{\lambda}) = \ln \mathcal{L}(N_{\rm obs}|\vec{\lambda}) + \sum^{N_{\rm obs}}_{i=1} \ln \mathcal{L}(D_i|\vec{\lambda}) \ .
\end{equation}







\section{Toy Model}

In this section, we delve into a toy-model problem of just using the \lvk\ chirp-masses and \compas\ chirp-masses, marginalising over the redshift. 

In this case, we create a surrogate for $\mathcal{L}$.





\section{Results}\label{sec:results}
Details on the results

\section{Discussion and caveats}\label{sec:conclusion}
Some concluding discussions

\section{Data and software availability}\label{sec:data}
Code data can be found at \projectGit



%%%%%%%%%%%%%%%%%%%%



\section*{Acknowledgments}{


We gratefully acknowledge the Swinburne Supercomputing OzSTAR Facility for computational resources. All analyses (including test and failed analyses) performed for this study used $XX$K core hours on OzSTAR. This would have amounted to a carbon footprint of ${\sim XX{\text{t CO}_2}}$~\citep{greenhouse, energy_to_co2_converter}. Thankfully, as OzSTAR is powered by wind energy from Iberdrola Australia; the electricity for computations produces negligible carbon waste.


A.V. is supported by the Australian Research Council (ARC) Centre of Excellence CE170100004.

}

\vspace{5mm}
\facilities{LIGO-VIRGO-KAGRA}

\software{
\python~\citep{pythonForScientificComputing,pythonForScientists},
\astropy~\citep{astropy},
\arviz~\citep{arviz_2019},
\exoplanet~\citep{Foreman-Mackey:2021:JOSS},
\lightkurve~\citep{LightkurveCollaboration:2018:ascl},
\starry~\citep{Luger:2019:AJ},
\celerite~\citep{Foreman-Mackey:2017:ascl},
\pymc~\citep{Salvatier:2016:ascl},
\numpy~\citep{numpy},
\scipy~\citep{SciPy},
\pandas~\citep{pandas},
\matplotlib~\citep{matplotlib},
\corner~\citep{corner},
\sphinx~\citep{sphinx_doc},
\jupyter~\citep{jupyter},
\jupyterbook~\citep{jupyter_book}.
}


% ADS bibliography link
% https://ui.adsabs.harvard.edu/user/libraries/_DyLS4HbTY-eJIMBiFUdxw
\bibliography{main}{}
\bibliographystyle{aasjournal}

%%%%%%%%%%%%%%%%%%%%%%%%%%%%%%%%%%%%%%%%%%%%
\appendix

\section{Some More dets}\label{apdx:dets}
Some additional dets useful for the reader



\end{document}
